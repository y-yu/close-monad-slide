\definecolor{links}{HTML}{2A1B81}
\hypersetup{colorlinks,linkcolor=,urlcolor=links}

\usetheme{Boadilla}
\usecolortheme{seahorse}
\usefonttheme{serif}
\beamertemplatenavigationsymbolsempty

\setbeamertemplate{bibliography item}{\insertbiblabel}
\setbeamersize{description width=1cm}

\usepackage{luacode}
\usepackage{luatexja}
\usepackage{pgfpages}
\usepackage[osf]{mathpazo}

\begin{luacode*}
  USE_IPAFONT = os.getenv"USE_IPAFONT"
  USE_YUFONT = os.getenv"USE_YUFONT"
  
  if USE_YUFONT == "true" then
    tex.sprint("\\AtBeginDocument{\\usepackage[yu-osx, deluxe, expert]{luatexja-preset}}")
  elseif USE_IPAFONT == "true" then
    tex.sprint("\\AtBeginDocument{\\usepackage[ipaex, deluxe, expert]{luatexja-preset}}")
  else
    tex.sprint("\\AtBeginDocument{\\usepackage[hiragino-pro, deluxe, expert]{luatexja-preset}}")
  end
\end{luacode*}

\usepackage{epigraph}
\usepackage{etoolbox}
\usepackage{tikz}
\usepackage{framed}
\usepackage{libertine}
\usepackage{amsmath}
\usepackage{mathtools}
\usepackage{listings}

\renewcommand{\kanjifamilydefault}{\gtdefault}

%\setbeameroption{show notes on second screen=right}

\setmainfont[Ligatures=TeX]{Linux Libertine O}
\setsansfont[Ligatures=TeX]{CMU Sans Serif}
\setmonofont[Ligatures=TeX]{CMU Typewriter Text}

\input{vc.tex}

\title[Close Monad]{%
  Close Monad\\
  {\normalsize \href{}{????????????}}
}
\author{吉村 優}
\date[May 24, 2017]{%
  May 24, 2017 \\%
  {\footnotesize (Commit ID: \GITAbrHash)}
}
\institute[]{%
   \href{mailto:yyu@mental.poker}{yyu@mental.poker}
}

\input{./lib/quotebox.tex}
\input{./lib/footnotemark.tex}
\input{./lib/ballon.tex}
\input{./lib/listings_setting.tex}

\newcommand\ballref[1]{%
\tikz \node[circle, shade,ball color=structure.fg,inner sep=0pt,%
  text width=8pt,font=\tiny,align=center] {\color{white}\ref{#1}};
}

\begin{document}

\frame{\maketitle}

\begin{frame}
  \frametitle{目次}

  \tableofcontents
\end{frame}

\section{自己紹介}
\begin{frame}
  \frametitle{自己紹介}
  
  \begin{columns}
    \begin{column}{0.4\textwidth}
      \centering
      \begin{figure}
        \includegraphics[width=0.95\textwidth]{img/bird2x.png}
      \end{figure}

      \begin{description}
        \item[Twitter] \href{https://twitter.com/\_yyu\_}{@\_yyu\_}
        \item[Qiita] \href{http://qiita.com/yyu}{yyu}
        \item[GitHub] \href{https://github.com/y-yu}{y-yu}
      \end{description}
    \end{column}
    \begin{column}{0.6\textwidth}
      \begin{itemize}
        \item<2-> 筑波大学 情報科学類卒(学士)
        \item<3-> 株式会社ドワンゴ 入社
        \item<4-> 基盤開発本部 共通基盤開発部 \\
        認証基盤セクション
        \item<5-> Scalaでアカウントシステムを開発中
      \end{itemize}
    \end{column}
  \end{columns}
\end{frame}

\section{モナドとは?}

\newcommand{\Point}[1]{\mathtt{point}\left(#1\right)}
\newcommand{\FlatMap}[2]{\mathtt{flatMap}\left(#1\right)\left(#2\right)}

\begin{frame}
  \frametitle{モナドとは?}

  \begin{itemize}
    \item<2-> 次の2つの関数\lstinline|point|/\lstinline|flatMap|を持つデータ構造
    \begin{itemize}
      \item<3-> \lstinline|point[A](a: A): M[A]|
      \item<3-> \lstinline|flatMap[A, B](a: M[A])(f: A => M[B]): M[B]|
    \end{itemize}
    \item<4-> この2つの関数は\textbf{モナド則}を満す
  \end{itemize}

  \uncover<5->{
    \begin{block}{モナド則}
      \begin{itemize}
        \item {\footnotesize $\FlatMap{\Point{x}}{f} \equiv f(x)$}
        \item {\footnotesize $\FlatMap{m}{\mathtt{point}} \equiv m$}
        \item {\footnotesize $\FlatMap{\FlatMap{m}{f}}{g} \equiv \FlatMap{m}{x \Rightarrow \FlatMap{f(x)}{g}}$}
      \end{itemize}
    \end{block}
  }

  \uncover<6->{
    \begin{exampleblock}{モナドの例}
      \begin{itemize}
        \item<7-> \lstinline|List[A]|
        \item<8-> \lstinline|Option[A]|
        \item<9-> \lstinline|Either[A, B]|
      \end{itemize}
    \end{exampleblock}
  }
\end{frame}

\begin{frame}
  \frametitle{モナドの例 ------ \protect\lstinline|Option[A]|}

  \begin{itemize}
    \item<2-> ある型\lstinline|A|の値があるかないかを表す
    \item<3-> 長さが最大で1のリストと考えることもできる
  \end{itemize}

  \uncover<4->{
    \lstinputlisting[style=scala]{./src/Option.scala}
  }
\end{frame}

\begin{frame}
  \begin{center}
    \uncover<2->{
      \begin{tikzpicture}
        \calloutquote[width=5cm,position={(0.7,-0.2)},fill=red!30,rounded corners]{いったい何が便利なの?}
      \end{tikzpicture}
    }

    \uncover<3->{
      \begin{tikzpicture}
        \calloutquote[width=10cm,position={(-0.7,-0.2)},fill=green!30,rounded corners]{モナドであれば\textbf{手続き的な処理}を関数合成で書ける!}
      \end{tikzpicture}
    }
  \end{center}
\end{frame}

\begin{frame}
  \frametitle{\protect\lstinline|Option|の手続き的な処理}

  \uncover<2->{
    \lstinputlisting[style=scala]{./src/ProceduralOption.scala}
  }

  \begin{itemize}
    \item<3-> \lstinline|isDefined|で\lstinline|None|か\lstinline|Some|かをチェックしたい
    \item<4-> \lstinline|if|がネストしていて見通しが悪い
  \end{itemize}
\end{frame}

\begin{frame}
  \frametitle{手続き的な処理}

  \uncover<2->{
    \lstinputlisting[style=scala]{./src/ProceduralClose.scala}
  }

  \begin{itemize}
    \item<3-> \lstinline|close|を忘れたらリソースリーク!
  \end{itemize}
\end{frame}

\section*{参考文献}
\begin{frame}
  \frametitle{参考文献}

  \bibliographystyle{junsrt_url}
  \nocite{*}
  \bibliography{ref}
\end{frame}

\begin{frame}
  \centering
  {\Huge Thank you for your attention!}
\end{frame}

\end{document}

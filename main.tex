\definecolor{links}{HTML}{2A1B81}
\hypersetup{colorlinks,linkcolor=,urlcolor=links}

\usetheme{Boadilla}
\usecolortheme{seahorse}
\usefonttheme{serif}
\beamertemplatenavigationsymbolsempty

\setbeamertemplate{bibliography item}{\insertbiblabel}
\setbeamersize{description width=1cm}

\usepackage{luacode}
\usepackage{luatexja}
\usepackage{pgfpages}

\begin{luacode*}
  USE_IPAFONT = os.getenv"USE_IPAFONT"
  USE_YUFONT = os.getenv"USE_YUFONT"
  
  if USE_YUFONT == "true" then
    tex.sprint("\\AtBeginDocument{\\usepackage[yu-osx, deluxe, expert]{luatexja-preset}}")
  elseif USE_IPAFONT == "true" then
    tex.sprint("\\AtBeginDocument{\\usepackage[ipaex, deluxe, expert]{luatexja-preset}}")
  else
    tex.sprint("\\AtBeginDocument{\\usepackage[hiragino-pro, deluxe, expert]{luatexja-preset}}")
  end
\end{luacode*}

\usepackage{epigraph}
\usepackage{etoolbox}
\usepackage{tikz}
\usepackage{framed}
\usepackage{libertine}
\usepackage{amsmath}
\usepackage{mathtools}
\usepackage{listings}
\usepackage{tikz-qtree}

\renewcommand{\kanjifamilydefault}{\gtdefault}

%\setbeameroption{show notes on second screen=right}

\setmainfont[Ligatures=TeX]{Linux Libertine O}
\setsansfont[Ligatures=TeX]{CMU Sans Serif}
\setmonofont[Ligatures=TeX]{CMU Typewriter Text}

\input{vc.tex}

\title[Close Monad]{%
  {\LARGE Close Monad}\\
  {\normalsize 作って学ぶモナド入門}
}
\author[吉村 優]{吉村 優 \\ Hikaru \textsc{Yoshimura}}
\date[May 24, 2017]{%
  May 24, 2017 \\%
  {\footnotesize (Commit ID: \GITAbrHash)}
}
\institute[株式会社ドワンゴ]{%
  株式会社ドワンゴ \\
   \href{mailto:yyu@mental.poker}{yyu@mental.poker}
}

\input{./lib/quotebox.tex}
\input{./lib/footnotemark.tex}
\input{./lib/ballon.tex}
\input{./lib/listings_setting.tex}

\newcommand\ballref[1]{%
\tikz \node[circle, shade,ball color=structure.fg,inner sep=0pt,%
  text width=8pt,font=\tiny,align=center] {\color{white}\ref{#1}};
}

\begin{document}

\frame{\maketitle}

\begin{frame}
  \frametitle{目次}

  \tableofcontents
\end{frame}

\section{自己紹介}
\begin{frame}
  \frametitle{自己紹介}
  
  \begin{columns}
    \begin{column}{0.3\textwidth}
      \centering
      \begin{figure}
        \includegraphics[width=0.95\textwidth]{img/bird2x.png}
      \end{figure}

      \begin{description}
        \item[Twitter] \href{https://twitter.com/\_yyu\_}{@\_yyu\_}
        \item[Qiita] \href{http://qiita.com/yyu}{yyu}
        \item[GitHub] \href{https://github.com/y-yu}{y-yu}
      \end{description}
    \end{column}
    \begin{column}{0.7\textwidth}
      \begin{itemize}
        \item<2-> 筑波大学 情報学群 情報科学類卒(学士)
        \item<3-> 株式会社ドワンゴ 入社
        \item<4-> 基盤開発本部 Dwango Cloud Service部\\
        認証基盤セクション
        \item<5-> Scalaでアカウントシステムを開発中
        \item<6-> 最近の趣味は量子コンピュータの勉強
      \end{itemize}
    \end{column}
  \end{columns}
\end{frame}

\section{モナドとは?}

\newcommand{\Point}[1]{\mathtt{point}\left(#1\right)}
\newcommand{\FlatMap}[2]{\mathtt{flatMap}\left(#1\right)\left(#2\right)}

\begin{frame}
  \frametitle{モナドとは?}

  \begin{itemize}
    \item<2-> 次の2つの関数\lstinline|point|/\lstinline|flatMap|を持つデータ構造
    \begin{itemize}
      \item<3-> \lstinline|point[A](a: A): M[A]|
      \item<3-> \lstinline|flatMap[A, B](a: M[A])(f: A => M[B]): M[B]|
    \end{itemize}
    \item<4-> この2つの関数は\textbf{モナド則}を満す
  \end{itemize}

  \uncover<5->{
    \begin{block}{モナド則\cite{monadlaws}}
      \begin{itemize}
        \item {\footnotesize $\FlatMap{\Point{x}}{f} \equiv f(x)$}
        \item {\footnotesize $\FlatMap{m}{\mathtt{point}} \equiv m$}
        \item {\footnotesize $\FlatMap{\FlatMap{m}{f}}{g} \equiv \FlatMap{m}{x \Rightarrow \FlatMap{f(x)}{g}}$}
      \end{itemize}
    \end{block}
  }

  \uncover<6->{
    \begin{exampleblock}{モナドの例}
      \begin{itemize}
        \item<7-> \lstinline|List[A]|
        \item<8-> \lstinline|IO[A]|
        \item<9-> \lstinline|Future[A]|
      \end{itemize}
    \end{exampleblock}
  }
\end{frame}

\begin{frame}
  \frametitle{モナドの例 --- \protect\lstinline|IO[A]|}

  \begin{itemize}
    \item<2-> IOを閉じ込めるデータ構造
  \end{itemize}

  \uncover<3->{
    \begin{exampleblock}{IOを利用するプログラム例}
      \lstinputlisting[style=scala]{./src/SideEffectIO.scala}
    \end{exampleblock}
  }

  \begin{itemize}
    \item<4-> \lstinline|readAndPrint|は\textbf{参照透過}ではない
  \end{itemize}

  \uncover<5->{
    \begin{block}{参照透過\cite{fpinscala}}
      \begin{shadequote}{}
        式$e$があり、プログラム$p$において$p$の意味に影響を与えることなく$p$内の$e$を
        全て$e$の評価結果に置き換えることができる
      \end{shadequote}
    \end{block}
  }
\end{frame}

\begin{frame}
  \frametitle{モナドの例 --- \protect\lstinline|IO[A]|}

  \lstinputlisting[style=scala]{./src/IO.scala}

  \begin{itemize}
    \item<2-> 副作用を\lstinline|run|メソッドに閉じ込める
    \item<3-> \lstinline|apply|は\lstinline|point|の別名
  \end{itemize}
\end{frame}

\begin{frame}
  \frametitle{モナドの例 --- \protect\lstinline|IO[A]|}

  \begin{itemize}
    \item<1-> \lstinline|IO|で副作用を隠蔽する
  \end{itemize}

  \uncover<2->{
    \lstinputlisting[style=scala]{./src/ReadAndPrintWithoutFor.scala}
  }

  \begin{itemize}
    \item<3-> \lstinline|readAndPrint|が参照透過になる
    \item<4-> 最後の\lstinline|run|メソッドで副作用を呼び出す
    \item<5-> \lstinline|map|/\lstinline|flatMap|がネストしていて見通しが悪い
  \end{itemize}
\end{frame}

\begin{frame}
  \frametitle{モナドの例 --- \protect\lstinline|IO[A]|}

  \lstinputlisting[style=scala]{./src/ReadAndPrint.scala}

  \begin{itemize}
    \item<2-> \lstinline|for|-\lstinline|yield|は
    \lstinline|map|/\lstinline|flatMap|のシンタックスシュガー
    \item<3-> \lstinline|map|/\lstinline|flatMap|のネストを解消
  \end{itemize}
\end{frame}

\begin{frame}
  \frametitle{ここまでのまとめ}

  \begin{itemize}
    \item<2-> 副作用をモナドに閉じ込めることができる
    \item<3-> モナドと\lstinline|for|-\lstinline|yield|で合成を手続き的に書ける
    \item<4-> ドワンゴ社内ではオリジナルのモナドも使われている
    \begin{itemize}
      \item<5-> Fujitask(トランザクションモナド)\cite{fujitask,scalamatsuri2016,fujitasksimple}
      \item<5-> Do(ロールバックモナド)
    \end{itemize}
  \end{itemize}

  \begin{center}
    \uncover<6->{
      \begin{tikzpicture}
        \calloutquote[width=11cm,position={(0.7,-0.2)},fill=blue!30,rounded corners]{\LARGE 自分でも何かモナドを作りたい!}
      \end{tikzpicture}
    }
  \end{center}  
\end{frame}

\section{リソースの解放とローンパターン}

\begin{frame}
  \frametitle{リソースを用いる手続き的な処理}

  \uncover<2->{
    \lstinputlisting[style=scala]{./src/ProceduralClose.scala}
  }
  
  \begin{itemize}
    \item<3-> リソース確保と解放が副作用
    \item<4-> \lstinline|close|を忘れたらリソースリーク!
  \end{itemize}
\end{frame}

\begin{frame}
  \frametitle{ローンパターン\cite{loanpattern}を用いた方法}

  \uncover<2->{
    \lstinputlisting[style=scala]{./src/Using.scala}
  }
  
  \begin{itemize}
    \item<3-> 関数\lstinline|process|とリソース\lstinline|resource|を受け取る
    \item<4-> \lstinline|process|が終了したら結果に関わらず\lstinline|close|を実行
    \item<5-> \lstinline|close|の呼び出しは自動的なので、忘れることがない
  \end{itemize}
\end{frame}

\begin{frame}
  \frametitle{ローンパターン\cite{loanpattern}を用いた方法}

  \begin{itemize}
    \item ただ、複数のリソースを用いるとネストする
  \end{itemize}

  \uncover<2->{
    \lstinputlisting[style=scala]{./src/NestedUsing.scala}
  }

  \begin{itemize}
    \item<3-> これは微妙すぎる……
  \end{itemize}
\end{frame}

\section{Close Moand}

\begin{frame}
  \centering
  {\Huge Close Monad}
\end{frame}

\begin{frame}
  \lstinputlisting[style=scala]{./src/Close.scala}
\end{frame}

\begin{frame}
  \frametitle{Close Monadの実装}

  \lstinputlisting[style=scala, firstline=1, lastline=9]{./src/Close.scala}

  \begin{itemize}
    \item<2-> 型パラメータ\lstinline|R|はリソースの型を表す
    \item<3-> 型パラメータ\lstinline|A|は結果の型を表す
    \item<4-> 型\lstinline|R|の引数\lstinline|res|を取る
    \item<5-> 抽象メソッド\lstinline|process|は型クラス\lstinline|Closer[R]|のインスタンスを使う
    \item<6-> メソッド\lstinline|run|は\lstinline|try-finally|内で\lstinline|process|を実行する
  \end{itemize}
\end{frame}

\begin{frame}
  \frametitle{Close Monadの合成}

  \lstinputlisting[style=scala, firstline=11, lastline=18]{./src/Close.scala}

  \begin{itemize}
    \item<2-> 型\lstinline|R|より抽象的な型\lstinline|AR|に対して、
    \lstinline|A => Close[AR, B]|となる関数\lstinline|f|を受け取る
    \item<3-> これらを用いて新しい\lstinline|Close[AR, B]|をインスタンス化
    \item<4-> \lstinline|process|は、\lstinline|try|内で
    \lstinline|self.process|を実行してその結果を\lstinline|f|に渡して、
    さらにその結果の\lstinline|process|を実行する
    \begin{itemize}
      \item<5-> 最終的に\lstinline|f(self.process()).process()|の結果の型は\lstinline|B|となる
    \end{itemize}
    \item<6-> \lstinline|finally|の中でインスタンス\lstinline|closer|を用いて
    リソースをクローズする
  \end{itemize}
\end{frame}

\begin{frame}
  \frametitle{Close Monadの合成}

  \lstinputlisting[style=scala, firstline=20, lastline=21]{./src/Close.scala}

  \begin{itemize}
    \item<2-> 一度でも合成された\lstinline|Close|の\lstinline|run|メソッドを上書きする
    \item<3-> 単に\lstinline|process|を呼び出すだけにする
    \begin{itemize}
      \item<4-> 未合成の\lstinline|Close|インスタンスの\lstinline|run|が呼ばれた場合でも
      リソースをクローズするため、最初の\lstinline|run|の定義では\lstinline|try-finally|で
      リソースをクローズしていた
      \item<5-> 合成された場合、\lstinline|flatMap|によって\lstinline|process|内で
      \lstinline|try-finally|が呼び出されるので\lstinline|run|では不要となる
    \end{itemize}
  \end{itemize}
\end{frame}

\begin{frame}
  \frametitle{Close Monadの合成}

  \lstinputlisting[style=scala, firstline=24, lastline=25]{./src/Close.scala}

  \lstinputlisting[style=scala]{./src/CloseObject.scala}

  \begin{itemize}
    \item<2-> \lstinline|map|は\lstinline|flatMap|と\lstinline|Close\#apply|を用いて定義
    \begin{itemize}
      \item<3-> \lstinline|apply|は\lstinline|point|と同じ役割
    \end{itemize}
    \item<4-> \lstinline|IO|とほぼ同様
  \end{itemize}
\end{frame}

\begin{frame}
  \frametitle{型クラス\protect\lstinline|Closer[A]|}

  \lstinputlisting[style=scala]{./src/Closer.scala}

  \begin{itemize}
    \item<2-> 基本的にはローンパターンの時と同じ
  \end{itemize}
\end{frame}

\begin{frame}
  \frametitle{Example}

  \lstinputlisting[style=scala]{./src/Main.scala}

  \begin{itemize}
    \item<2-> \lstinline|for-yield|で簡潔に書ける
  \end{itemize}
\end{frame}

\section{テスト}

\begin{frame}
  \begin{center}
    \uncover<1->{
      \begin{tikzpicture}
        \calloutquote[width=8cm,position={(0.7,-0.2)},fill=red!30,rounded corners]{このモナドはモナド則を満しているのか?}
      \end{tikzpicture}
    }

    \uncover<2->{
      \begin{tikzpicture}
        \calloutquote[width=8cm,position={(-0.7,-0.2)},fill=green!30,rounded corners]{\textbf{プロパティベーステスト}で確かめよう!}
      \end{tikzpicture}
    }
  \end{center}  
\end{frame}

\begin{frame}
  \frametitle{プロパティベーステスト}

  \uncover<2->{
    \begin{shadequote}{}
      テスト仕様(property)を元にテストデータを自動生成し、
      ひとつのテスト仕様に多数のテストデータを適用するテストのこと
    \end{shadequote}
  }

  \begin{itemize}
    \item<3-> 今回はScalaのプロパティベーステストツールであるscalaprops\cite{scalaprops}を採用
    \begin{itemize}
      \item<4-> Scalaz\cite{scalaz}のテストにも採用されている
    \end{itemize}
    \item<5-> 証明とは違うので完全ではないが比較的簡単
  \end{itemize}
\end{frame}

\begin{frame}
  \frametitle{scalaprops}

  \begin{itemize}
    \item<2-> モナド則のプロパティはすでに用意されている
    \item<3-> Close Monadのインスタンスを自動生成する部分を作ればいい
    \item<4-> \lstinline|Gen[A]|という型クラスのインスタンスで
    型\lstinline|A|の生成方法を定義する
    \item<5-> テストでは比較をするので、型\lstinline|A|を比較するための
    型クラス\lstinline|Equal[A]|のインスタンスも定義していく
  \end{itemize}
\end{frame}

\begin{frame}
  \frametitle{\protect\lstinline|Gen|と\protect\lstinline|Equal|の作成}

  \uncover<2->{
    \lstinputlisting[style=scala, firstline=1, lastline=8]{./src/CloseTestHelper.scala}
  }

  \begin{itemize}
    \item<3-> 例外を送出することを想定して、テスト専用の例外\lstinline|Error|と
    それの生成方法\lstinline|Gen[Error]|を定義する
    \item<4-> 同じ例外が送出されることも重要なので、
    比較できるように\lstinline|Equal[Error]|も定義する
  \end{itemize}
\end{frame}

\begin{frame}
  \frametitle{\protect\lstinline|Gen|と\protect\lstinline|Equal|の作成}

  \lstinputlisting[style=scala, firstline=10, lastline=19]{./src/CloseTestHelper.scala}
  \lstinputlisting[style=scala]{./src/TestCloseable.scala}

  \begin{itemize}
    \item<2-> リソースの型である\lstinline|Closeable|もテスト専用のものを定義
    \item<3-> 割合で別のインスタンスが生成されるようにする
  \end{itemize}
\end{frame}

\begin{frame}
  \frametitle{\protect\lstinline|Gen|と\protect\lstinline|Equal|の作成}

  \lstinputlisting[style=scala, firstline=27, lastline=29]{./src/CloseTestHelper.scala}
  \lstinputlisting[style=scala]{./src/TestCloser.scala}

  \begin{itemize}
    \item<2-> テスト用の\lstinline|Closer|には開放されたリソースを格納する配列を用意
    \item<3-> この配列で正しい順番でリソースがクローズされているのかをテストできる
  \end{itemize}
\end{frame}

\begin{frame}
  \frametitle{\protect\lstinline|Gen|と\protect\lstinline|Equal|の作成}

  \lstinputlisting[style=scala, firstline=31, lastline=46]{./src/CloseTestHelper.scala}

  \begin{itemize}
    \item<2-> \lstinline|run|の結果が同じかつリソースが同じ順序でクローズされていることを
    同一性の条件にする
  \end{itemize}
\end{frame}

\begin{frame}
  \frametitle{\protect\lstinline|Gen|と\protect\lstinline|Equal|の作成}

  \lstinputlisting[style=scala, firstline=48]{./src/CloseTestHelper.scala}

  \begin{itemize}
    \item<2-> 今まで作った\lstinline|Gen|を用いて\lstinline|Gen[Close[R, A]]|を定義
    \item<3-> 適当な割合で例外をスローする\lstinline|Close|と正常な値を返す\lstinline|Close|を生成
  \end{itemize}
\end{frame}

\begin{frame}
  \frametitle{\protect\lstinline|Gen|と\protect\lstinline|Equal|の作成}

  \begin{itemize}
    \item<1-> \lstinline|Gen|を図にすると次のようになる
    \begin{center}
      \uncover<2->{
        \begin{tikzpicture}[scale=.7]
          \tikzset{level distance=1.5cm, grow'=up}
          \Tree[.\node{\lstinline|Gen[Close[R, A]]|};
          \node{\lstinline|Gen[A]|};
          [.\node{\lstinline|Gen[Error]|}; \node{\lstinline|Gen[Int]|}; ]
          \node{\lstinline|Gen[R]|}; ]
        \end{tikzpicture}
      }
    \end{center}

    \item<3-> \lstinline|Equal|を図にすると次のようになる
    \begin{center}
      \uncover<4->{
        \begin{tikzpicture}[scale=.7]
          \tikzset{level distance=1.5cm, grow'=up}
          \Tree[.\node{\lstinline|Equal[Close[R, A]]|};
          \node{\lstinline|Equal[A]|};
          [.\node{\lstinline|Equal[Error]|}; \node{\lstinline|Equal[Int]|}; ]
          [.\node{\lstinline|Gen[TestCloser[R]]|}; \node{\lstinline|Gen[R]|}; ]
          [.\node{\lstinline|Equal[ArrayBuffer[R]]|}; \node{\lstinline|Equal[R]|}; ] ]
        \end{tikzpicture}
      }
    \end{center}
  \end{itemize}
\end{frame}

\begin{frame}
  \frametitle{テスト}

  \lstinputlisting[style=scala]{./src/CloseTest.scala}

  \begin{itemize}
    \item<2-> Scalazのモナドのインスタンスを作成
    \item<3-> scalapropsに用意されているモナドの性質をテストするケースを投入
  \end{itemize}
\end{frame}

\section{まとめ}

\begin{frame}
  \frametitle{まとめ}

  \begin{itemize}
    \item<2-> リソース管理をモナドにした
    \item<3-> モナド則を満しているかどうかをプロパティベーステストで検証した
    \begin{itemize}
      \item<4-> ただ遅延評価と組み合わせても安全なのかは疑問が残る
      \item<5-> プロダクションで利用するにはやや不安
    \end{itemize}
    \item<6-> Close MonadについてはQiitaにも書いてGitHubに置いた\cite{closemonad,closemonadgithub}
    \item<7-> 継続モナドを用いたリソース管理の方法もある\cite{tanakh2015,jwhaco2017}
    \begin{itemize}
      \item 継続モナドを利用すると全てのローンパターンをモナドにできるかも
    \end{itemize}
  \end{itemize}
\end{frame}

\section{参考文献}

\begin{frame}[allowframebreaks]
  \frametitle{参考文献}

  \bibliographystyle{junsrt_url}
  \nocite{*}

  \bibliography{ref}
\end{frame}

\begin{frame}
  \centering
  {\Huge Thank you for your attention!\\
    \vspace{1em}
    Any question?
  }
\end{frame}

\end{document}

\section*{予備スライド}

\begin{frame}
  \frametitle{モナドの例 --- \protect\lstinline|Option[A]|}

  \begin{itemize}
    \item<2-> ある型\lstinline|A|の値があるかないかを表す
    \item<3-> 長さが最大で1のリストと考えることもできる
  \end{itemize}

  \uncover<4->{
    \lstinputlisting[style=scala]{./src/Option.scala}
  }

  \begin{itemize}
    \item<5-> \lstinline|point|と\lstinline|flatMap|から\lstinline|map|も定義
  \end{itemize}
\end{frame}

\begin{frame}
  \begin{center}
    \uncover<1->{
      \begin{tikzpicture}
        \calloutquote[width=5cm,position={(0.7,-0.2)},fill=red!30,rounded corners]{いったい何が便利なの?}
      \end{tikzpicture}
    }

    \uncover<2->{
      \begin{tikzpicture}
        \calloutquote[width=10cm,position={(-0.7,-0.2)},fill=green!30,rounded corners]{モナドであれば\textbf{手続き的な処理}を関数合成で書ける!}
      \end{tikzpicture}
    }
  \end{center}
\end{frame}

\begin{frame}
  \frametitle{\protect\lstinline|Option|の手続き的な処理}

  \uncover<2->{
    \lstinputlisting[style=scala]{./src/ProceduralOption.scala}
  }

  \begin{itemize}
    \item<3-> \lstinline|isDefined|で\lstinline|None|か\lstinline|Some|かをチェックしたい
    \item<4-> \lstinline|if|がネストしていて見通しが悪い
  \end{itemize}
\end{frame}

\begin{frame}
  \frametitle{\protect\lstinline|Option|の\protect\lstinline|flatMap|/\lstinline|map|を用いた処理}

  \uncover<2->{
    \lstinputlisting[style=scala]{./src/FlatMapOption.scala}
  }

  \begin{itemize}
    \item<3-> \lstinline|if|のネストを解消
    \item<4-> ただ\lstinline|flatMap|/\lstinline|map|がネストしている
  \end{itemize}
\end{frame}

\begin{frame}
  \frametitle{\protect\lstinline|Option|の\protect\lstinline|for|を用いた処理}

  \uncover<2->{
    \lstinputlisting[style=scala]{./src/ForOption.scala}
  }

  \begin{itemize}
    \item<3-> \lstinline|for|-\lstinline|yield|は\lstinline|flatMap|/\lstinline|map|のシンタックスシュガー
    \item<4-> ネストなどを排除した簡潔なコードに
  \end{itemize}
\end{frame}


